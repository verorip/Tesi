\chapter*{Abstract}
\addcontentsline{toc}{chapter}{Abstract}

La tesi affronta alcune problematiche presenti al giorno d'oggi in molti progetti informatici,
dovute al continuo sviluppo di metodologie di interfacciamento sempre pi\`u
dirette e complesse.
Il software nel quale sono state affrontate queste problematiche \`e il MagicMirror, un progetto di domotica.\\
In particolare il problema \`e stato implementare metodi alternativi di interfacciamento, rispetto a mouse e tastiera,
e metodi per configurare il software da remoto da parte di utenti non esperti.\\[1\baselineskip]
Per affrontare il primo problema sono state implementate due interfacce utente: una vocale, per permettere
l'impartizione di comandi tramite specifiche frasi, e una touch, per eseguire specifiche operazioni
tramite movimenti delle dita su un'area di un dispositivo.
Per il secondo problema \`e stata sviluppata un'interfaccia web di configurazione remota che permettesse di
modificare facilmente le impostazioni del software principale.\\[1\baselineskip]
Per le interfacce utente di controllo sono stati utilizzati servizi esterni
all'ambiente in cui \`e stato sviluppato il progetto. Per l'interfaccia web di configurazione, invece, sono state usate tecnologie
per la maggior parte gi\`a implementate nel software principale.
