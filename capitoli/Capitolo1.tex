\chapter{Scopi e Problemi Affrontati}

\section{Moduli del MagicMirror}
Nello sviluppo dei moduli del MagicMirror si \'e dovuto affrontare il problema di far comunicare
i diversi dispositivi hardware (tra cui microfono e telecamera) con il calcolatore principale, utilizzando tecnologie,
software e linguaggi diversi tra loro, scelti in base alle loro caratteristiche e ai loro punti di forza.
Inoltre era necessario che i moduli comunicassero tra di loro perchè si potessero coordinare, o sempicemente
perchè un modulo, in base a degli input ricevuti, doveva controllarne un altro.
Un'altro problema rilevante era la corretta scelta e gestione degli input, essendo che sia con il microfono sia
con la telecamera c'erano più soluzioni di acquisizione, che potevano essere in tempo reale (stream)
o tramite una registrazione avvenuta precedentemente.\\
Lo scopo dei moduli \'e di permettere l'interazione tra il software principale
e l'essere umano, e di poter controllare le funzioni della macchina senza dover utilizzare
i tradizionali metodi di input (mouse e tastiera), e accedere ai diversi servizi che si possono implementare.
\\[2\baselineskip]
\section{Interfaccia di controllo del MagicMirror}
Nello sviluppo della pagina Web di controllo si \'e affrontato il problema di dover
implementare un'altrazione che permettesse di modificare
il documento di configurazione del software principale da un'interfaccia senza eliminarne parti essenziali
o modificarlo in un formato sbagliato (gestito tramite validatore).
Lo scambio dei messaggi tra il Backend ed il Frontend avviene in formato JSON, lo stesso formato
con cui \'e stipulato il file di configurazione del Magic Mirror e dei suoi moduli.\\
Lo scopo dell'interfaccia è di permettere ad un qualsiasi utente (con i permessi) di gestire
i diversi moduli implementati nel software modificandone il JSON, senza accedere fisicamente alla macchina.
La pagina, inoltre, permette di configurare moduli creati anche da terzi: inserendoli
semplicemente nella directory del dispositivo il Backend li cerca e li indicizza nell'interfaccia,
da dove possono essere attivati.
\\[2\baselineskip]
Nei cpaitoli successivi verrà spiegata la struttura del Magic Mirror 
