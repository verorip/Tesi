\chapter{Nozioni, linguaggio e strumenti utilizzati}
Durante il percorso di stage presso Wellnet ho appreso conoscenze riguardanti
le tecnologie più recenti e promettenti nel campo dei sistemi e delle
macchine virtuali (Docker \cite{docker-website}), utilizzando il nuovo linguaggio di programmazione
Go (Golang \cite{go-website}) sviluppato da Google e tecnologie di versioning del codice stabili
e collaudate (Git \cite{git-website}).
\\
Nel corso dello sviluppo del tool abbiamo cercato di seguire una metodologia di
lavoro chiamata Agile, che descrive una serie di principi che aiutano lo sviluppatore
ed il team di sviluppo nel produrre software di qualità.

\section{Docker}
Docker \cite{docker-website} è un software open source per la gestione dei container:
 aiuta ad amministrare in maniera più semplice i Containers Linux (tramite
 prima i driver LXC, poi recentemente sostituiti da runC \cite{runC-website}).
Lo scopo di Docker è quello di fornire un layer di astrazione intermedio (tra
la macchina e l'applicazione) che fornisce alle applicazioni un kernel
linux condiviso.
Insieme alla gestione del kernel condiviso Docker offre anche un sistema di
distribuzione delle immagini Docker (modelli di containers) e numerosi tool
per una migliore gestione dei container (dalla gestione dei cluster di containers
con Docker Swarm a interfacce grafiche per l'hub di distribuzione).
\\
Docker permette di gestire il meccanismo di isolamento di processi in un kernel
condiviso. È possibile isolare in modo semplice un singolo processo (quindi
una singola applicazione) in un ambiente protetto e gestibile, sia dal punto
di vista della sicurezza, che del consumo delle risorse.

\subsection{Differenza tra l’uso di Docker e le Virtual Machines}
Negli ultimi anni è cresciuto l'interesse per l'utilizzo dei container
linux come sostituti delle macchine virtuali.
Mentre una virtual machine offre una virtualizzazione a livello hardware
(full virtualization), i container linux offrono una virtualizzazione a livello
del sistema operativo (OS virtualization).
\\[2\baselineskip]
Una virtualizzazione a livello hardware fornisce un isolamento maggiore (ad esempio
la comunicazione tra virtual machines deve avvenire attraverso la rete) ma
porta con sè alcuni difetti, tra i quali grandi costo a livello hardware e
una gestione complicata delle macchine\cite{dockerAndContainers-slides}.\\
Una virtualizzazione a livello di sistema operativo presenta numerosi punti di forza:

\begin{itemize}
\item il supporto da parte di qualsiasi kernel linux abbastanza recente (e quindi la possibilità di essere usato su dispositivi molto vari, da un server ad un sistema embedded)
\item la velocità, sia in avvio che in esecuzione di applicazioni \cite{containerVSvm-paper}
\item la flessibilità e l'efficienza nell'uso delle risorse: in caso di necessità è possibile replicare il container e scalare orizzontalmente l'applicazione
\item un isolamento \textquotedblleft{}diverso\textquotedblright{}: non forte come quello delle virtual machines, ma permette ai container di comunicare anche condividendo files, socket, code FIFO\ldots{} proprio perché hanno un kernel condiviso. \cite{linuxContainers-paper}
\item la distribuzione dei modelli di containers: una immagine di Docker è di dimensione molto ridotta rispetto all'immagine di un sistema operativo per una virtual machine: è più facile distribuire una immagine.
\end{itemize}

Oltre a queste caratteristiche dei containers linux, Docker fornisce nativamente
un protocollo per la distribuzione delle immagini (modelli di containers)
pubblico e open-source.



\section{Go}
Go è un linguaggio di programmazione open source sviluppato inizialmente in
Google nel 2009 da Robert Griesemer (programmatore), Rob Pike (informatico, creatore
insieme a Ken Thompson dello standard UTF-8) e Ken Thompson (informatico, creatore
del linguaggio B e vincitore del premio Turing nel 1983).
Il linguaggio Go è compilato, fortemente tipizzato, concorrente e ha un
sistema di garbage collection per la gestione della memoria\cite{go-website}.

I programmi sono composti da packages i quali permettono una gestione efficiente
delle dipendenze. L'implementazione attuale utilizza il tradizionale modello
compilazione/link per generare i file eseguibili.
La grammatica è compatta e regolare e si focalizza sull'uso di poche
keywords (25 in totale) dalla forte chiarezza semantica. \cite{goSpec-website}
Il codice sorgente è codificato nativamente secondo lo standard Unicode UTF-8.
\\[2\baselineskip]
Una delle caratteristiche che hanno permesso al linguaggio Go di acquisire
un sempre maggior successo \cite{programmingLanguagesIndex-website} è l'uso di alcuni tool integrati nel programma
di sviluppo che ne permettono una scrittura rapida e sicura.
Ne sono un esempio gofmt e golint, che permettono di avere uno stile univoco nel
codice e uno standard nell'uso di spazi e indentazioni, e godoc e go get per una
generazione standard di documentazione e gestione delle librerie.


\section{Git}
Git\cite{git-website} è un software per la gestione del versionamento del codice creato da Linus Torvalds nel 2005.
Semplificando, git non è altro che un grafo diretto aciclico (DAG) di oggetti,
ciascuno identificato da un hash univoco\cite{gitCS-website}.\\
Gli oggetti presenti nel grafo sono principalmente tre: i blob, gli alberi e le etichette.
I blob sono semplici insiemi di bytes che contengono i dati contenuti nei files,
gli alberi rappresentano le directory mentre le etichette mostrano dati aggiuntivi
per identificare e gestire i nodi.\\
Quando un nodo punta ad un altro nodo viene definita una relazione di dipendenza tra i due:
 il secondo non può esistere senza il primo.
Le relazioni tra i singoli nodi (blob o alberi) e le etichette tra i nodi definiscono
la struttura del progetto e la sua storia (il suo stato prima e dopo ogni singolo cambiamento).
Questa struttura permette di visualizzare e gestire integramente tutte le modifiche avvenute su un file.
\\[1\baselineskip]
Due caratteristiche che hanno reso git (o sue alternative per il versionamento del codice)
uno strumento indispensabile per ogni sviluppatore: la semplicità di utilizzo e
la gestione distribuita tramite repository online.
Semplicità di utilizzo perché per iniziare ad usare git in maniera
efficace, anche grazie ai tool grafici presenti in rete, non è necessaria
alcuna conoscenza pregressa.
Un repository online (nel nostro caso abbiamo usato bitbucket per lo sviluppo
iniziale e poi github) permette di condividere il codice scritto tra i componenti
del team e gestire lo stato di ogni file sorgente tra i partecipanti al repository.



\section{Metodologie agili}
Per metodologie agili si intendono una serie di principi (o modelli) per lo
sviluppo software attraverso i quali il prodotto prende forma e si evolve
attraverso la collaborazione  tra il team di sviluppo ed il cliente.
Il termine \textquotedblleft{}agile\textquotedblright{} venne utilizzato
 per la prima volta nel 2001 nel \textquotedblleft{}manifesto Agile\textquotedblright{},
 anche se lo sviluppo di tali metodologie è iniziato già nella metà
 degli anni '90 in contrapposizione ai metodi a cascata tradizionali.
\\[1\baselineskip]
\begin{center}
  \textbf{Manifesto agile} \cite{agileManifest-website}
\\[1\baselineskip]
Stiamo scoprendo modi migliori di creare software,
sviluppandolo e aiutando gli altri a fare lo stesso.
Grazie a questa attività siamo arrivati a considerare importanti:
\\[1\baselineskip]

\textbf{Gli individui e le interazioni} più che \textit{i processi e gli strumenti}\\
\textbf{Il software funzionante} più che \textit{la documentazione esaustiva}\\
\textbf{La collaborazione col cliente} più che \textit{la negoziazione dei contratti}\\
\textbf{Rispondere al cambiamento} più che \textit{seguire un piano}
\\[1\baselineskip]

Ovvero, fermo restando il valore delle voci a \textit{destra},\\
consideriamo più importanti le voci a \textbf{sinistra}.
\end{center}
\newpage
Poiché nel manifesto agile sono descritti solamente i principi sui quali è
fondata la metodologia agile, nel corso degli anni sono state date differenti
definizioni di Metodologie o Metodi Agili, tra le quali una in particolare le
descrive in modo sufficientemente completo:\\
Il metodo agile è un modello di sviluppo software incentrato sulle persone,
orientato alla comunicazione, flessibile e reattivo (in modo tale da adattarsi
in ogni momento a cambiamenti voluti e non), veloce (incoraggiando uno sviluppo
del prodotto rapido ed iterativo), snello (focalizzandosi su piccoli intervalli
di tempo e sulla qualità) e proattivo (puntando sul miglioramento durante e
dopo lo sviluppo)\cite{agileMethod-website}.\\
Su questo modello sono nate numerose metodologie specifiche e spesso si parla
di \textquotedblleft{}tecniche agili\textquotedblright{}. \\
Le più popolari sono: Scrum, Kanban, Adaptive Software Development, Extreme Programming, Lean Software Development, etc.\\
Durante lo sviluppo, vista la natura dinamica del progetto, abbiamo utilizzato
principalmente un mix di queste \textquotedblleft{}tecniche agili\textquotedblright{}:
Adaptive Software Development (ASD), Scrum e Kanban.

\subsection{Adaptive Software Development (ASD)}
Lo sviluppo Adaptive Software Development promuove il paradigma adattivo ed è
stato proposto per la prima volta da Jim Highsmith nel 1999 nel suo libro "Adaptive Software Development".
Lo sviluppo secondo questo paradigma si focalizza sulla collaborazione tra gli
individui e l'organizzazione autonoma del team.\\
A differenza di altre \textquotedblleft{}tecniche agili\textquotedblright{}
si basa su un ciclo diviso in tre fasi: una di speculazione, una di
collaborazione ed una di apprendimento.\\
Questo ciclo permette uno sviluppo continuativo, dinamico e adattivo del
progetto e  favorisce una maggiore tolleranza ai cambiamenti.
Il termine \textquotedblleft{}speculazione\textquotedblright{} prende il posto
di \textquotedblleft{}pianificazione\textquotedblright{} di un normale ciclo di
vita del software e si riferisce al paradosso della pianificazione dettagliata:
nel caso di progetti basati su tecnologie molto recenti ed in continuo mutamento
 (ad esempio Docker: che durante i mesi di stage ha sviluppato una versione
 nativa per macchine non linux ed una gestione di pacchetti di containers sperimentale)
  è necessario che alcune fasi della pianificazione siano continuamente
  messi in discussione ed aggiornate.\\
Per collaborazione ed apprendimento ci si riferisce allo sforzo di gestire la
speculazione tra i membri del team ed alla risposta dello stesso per modificare
e adattare lo sviluppo alle nuove sfide.\\

\subsection{Scrum}
Scrum è un metodo di sviluppo agile pensato all'inizio degli anni '90 da
Jeff Sutherland. Nel corso degli ultimi anni lo studio e lo sviluppo di questo
metodo è stato portato avanti da numerose personalità di spicco dell'ingegneria
del software, tra cui Schwaber e Beedle.\\
Il metodo Scrum si basa sui princìpi del manifesto agile e tali princpipi sono
usati per guidare il ciclo di vita del prodotto, suddiviso in cinque principali
attività: requisiti, analisi, progettazione, evoluzione e consegna.
Ogni attività di lavoro segue un particolare modello temporale chiamato
\textquotedblleft{}sprint\textquotedblright{}.

Scrum si focalizza su un insieme di metodologie dalla provata efficacia per
progetti nei quali sono presenti particolari caratteristiche: tempistiche
ristrette e requisiti dinamici.
Come si può vedere nell'immagine, il metodo definisce alcune strutture utili:\\
\textbf{Backlog}: una lista delle priorità nei requisiti del progetto o caratteristiche
che potrebbero dare maggior valore al prodotto.
Il project manager prepara il backlog ed aggiorna la lista.\\
\textbf{Sprints}: uno sprint consiste in una unità di base dello sviluppo
secondo il metodo Scrum. è di durata fissa (da una a quattro settimane) e
durante lo sprint vengono create porzioni complete del prodotto.
Durante lo sprint non è possibile cambiare  gli obiettivi ma è necessario
aspettare il termine dello stesso per eventuali modifiche.\\
Scrum incorpora inoltre una serie di caratteristiche che enfatizzano le
priorità del progetto, la suddivisione del progetto in blocchi, la comunicazione
tra il team e un feedback continuo con il cliente.\\
Nel corso dello stage abbiamo definito sprint di 15 giorni per tutta la durata dello sviluppo.
Ad ogni termine di ogni sprint il backlog è stato aggiornato con le
caratteristiche da implementare e sono state discusse eventuali problematicità
o scelte architetturali del tool.


\subsection{Kanban}
Kanban è una metodologia per gestione della conoscenza ispirata ai principi
dello sviluppo software agile, proposta da David J. Anderson nel 2007 ed
ufficializzata nel  libro \textquotedblleft{}Kanban\textquotedblright{}
dello stesso nel 2010.\\
Tale metodologia nasce però molto prima, negli anni '40, sulla base dei primi
processi di ottimizzazione negli stabilimenti Toyota per gestire le scorte dei
prodotti utilizzati nelle sue fabbriche. Per gestire al meglio gli inventari,
in relazione ai materiali consumati dai vari gruppi di lavoro, venne istituita
una tessera (un \textquotedblleft{}kanban\textquotedblright{}) che doveva essere
scambiata tra gli utilizzatori delle risorse\cite{kanban-website}.\\
Quando un materiale finiva, un kanban veniva passato dalla fabbrica al magazzino
con tutti i dati necessari (tipo di materiale, esatta quantità del materiale
richiesto, etc). Il magazzino a sua volta riforniva la fabbrica e forniva un kanban al fornitore.\\
Mentre la tecnologia per rappresentare questo tipo di scambio si è evoluta,
questo processo produttivo è rimasto pressoché identico.\\
Oggi i team che utilizzano la metodologia Agile possono utilizzare gli stessi
principi sostituendo i materiali con le attività di sviluppo ed il passaggio di
consegne con il ciclo di vita del software.
Questo permette al team una migliore e più veloce pianificazione delle attività
ed una completa trasparenza nel ciclo di sviluppo.\\
In particolare, il metodo Kanban aiuta nella consapevolezza del team dello stato
del progetto, nella visualizzazione del flusso di lavoro e nella trasparenza delle
politiche di processo.\\
Nello sviluppo software di solito è utilizzata anche una Kanban board (o lavagna)
che raccoglie e suddivide le tessere Kanban in liste.

Durante tutta la durata dello stage abbiamo utilizzato una Kanban Board, con
tessere Kanban virtuali, chiamata Trello.\\
I progetti sono rappresentati da una lavagna, che contiene differenti liste
(stati del progetto), ciascuna contenenti le tessere Kanban (task del progetto).
Ad ogni completamento di una attività, la stessa viene spostata da una lista
all’altra. Ogni tessera (attività) contiene tutte le informazioni della stessa
attività (durata, descrizione, stato, se bloccante, etichette, etc.)\\
Le tessere progrediscono da una lista all’altra tramite un semplice drag-and-drop
 è possibile assegnare particolari notifiche alle liste o alle tessere.
