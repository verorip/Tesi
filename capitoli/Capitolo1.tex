\chapter{Scopi e Problemi}

\section{Moduli del MagicMirror}
Nello sviluppo dei moduli del MagicMirror si \'e dovuto affrontare il problema di far comunicare
i diversi dispositivi hardware (tra cui microfono e telecamera) con il calcolatore principale, utilizzando tecnologie
e software diversi tra loro e linguaggi differenti, dal momento che ognuno era risultato
più facile da implementare con uno specifico linguaggio rispetto ad un altro.\\
Lo scopo dei moduli \'e di permettere un'interazione umana tra il software principale
e l'essere umano, e di poter controllare la macchina senza bisogno di usare input meccanici come mouse e tastiera.
\\[2\baselineskip]
\section{Interfaccia di controllo del MagicMirror}
Nello sviluppo della pagina Web di controllo si \'e affrontato il problema di dover modificare
il documento di configurazione del software principale da una pagina web senza eliminarne parti essenziali
o modificarlo in un formato sbagliato (gestito tramite validatore).
Lo scambio dei messaggi tra il Backend ed il Frontend avviene in formato JSON, lo stesso formato
con cui \'e stipulato il file di configurazione del Magic Mirror.\\
Lo scopo dell'interfaccia è di permettere ad un qualsiasi utente (con i permessi) di gestire
i diversi moduli implementati nel software modificandone il JSON, senza accedere fisicamente alla macchina.
La pagina, inoltre, permette di configurare moduli creati anche da terzi: inserendoli
semplicemente nella directory del dispositivo il Backend li cerca e li indicizza nell'interfaccia,
da dove possono essere attivati.

L'ambiente in cui \'e stato svolto lo stage \'e Raspian una distribuzione Debian che gira
sul calcolatore Rasperry.
