\chapter{Tecnologie Implicate}

Durante lo sviluppo del progetto sono state adottate diverse tecnologie recenti
e non nel campo della creazione di moduli per un'applicazione(Electron \cite{Electron-website},
OpenCV \cite{OpenCV-website}, GoogleSpeechRecognition \cite{GoogleSTT-website})
e di web server (NodeJS \cite{NodeJS-website},  Model-View-Controller Architecture \cite{MVC-Architecture},
Express \cite{Express-website}, MySQL \cite{MySQL}, Mustache \cite{Mustache}),
utilizzando diversi linguaggi di programmazione (JavaScript \cite{JavaScript}, Python \cite{Python})
e tecnologie per visionare e condividere codici (GitLab \cite{git-website}).

\section{Electron}
Electron è un Framework open source rilasciato per la prima volta nel 2013, ma la prima versione
stabile è uscita di recente. \'E disponibile sui sistemi operativi Window, MacOS e Linux ed è scritto
in C++ e Javascript. Il framework permette la creazione di interfacce grafiche (GUI) per
applicazioni cross platform, utilizzando teconologie già esistenti per lo sviluppo del backend e del frontend
(Javascript, NodeJS, V8 \cite{V8}).
Un'applicazione Electorn ha bisongo di 3 componenti principali:
\begin{itemize}
\item Il package.json, un file JSON, che deve contenere almeno il nome dell'applicazione,
la versione dell'applicazione creata, la descrizione di quest'ultima e il il nome del file principale dell'applicazione
(necessaria per l'avvio)
\begin{lstlisting}
{
  "name": "magicmirror",
  "version": "2.1.1",
  "description": "The open source modular smart mirror platform.",
  "main": "js/electron.js"
}
\end{lstlisting}
\item fdsfaf
\item la flessibilità e l'ef
\end{itemize}
