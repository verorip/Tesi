\chapter{Tecnologie Implicate}

Lo sviluppo del progetto \'e stato svolto nell'ambiente Raspbian\cite{Raspbian}
una distribuzione Debian\cite{Debian} che gira sul dispositivo Rasperry\cite{Raspberry}.
Inoltre sono state adottate diverse tecnologie, recenti
e non, nel campo della creazione di un'applicazione e i relativi moduli(Electron \cite{Electron-website},
OpenCV \cite{OpenCV-website}, GoogleSpeechRecognition \cite{GoogleSTT-website})
e di web server (NodeJS \cite{NodeJS-website},  Model-View-Controller Architecture \cite{MVC-Architecture},
Express \cite{Express-website}, MySQL \cite{MySQL}, Mustache \cite{Mustache}),
diversi linguaggi di programmazione (JavaScript \cite{JavaScript}, Python \cite{Python})
e tecnologie per visionare e condividere codici (GitLab \cite{git-website}).

\section{Raspberry}
RaspberryPi \'e un calcolatore elettronico, montato su una singola scheda elettornica, a basso costo
dal consumo ridotto e alta portabilit\'a.
Rilasciato per la prima volta intorno al 2012 \'e diventato un prodotto utilizzato per una moltitudine
di progetti sia aziendali che casalinghi.
Il modello usato nello stage \'e RaspberryPi 3 model B e monta:
\begin{itemize}
\item una porta HDMI
\item porta LAN
\item uscita Aux
\item 4 porte USB
\item 40 General Purpose Input/Output(GPIO)
\item Scheda di rete wirless
\item Alimentazione microUSB 5V
\item un bus camera serial interface(CSI)
\item ingesso per microSD
\end{itemize}
Il sistema operativo per Raspberry deve essere installato su una miscroSD opportunamente formattata
e configurata con il corretto Master Boot Record (MBR).

\section{Raspbian}
Raspbian \'e un sistema operativo multi-architettura della distribuzione Debian, completamente libero,
ottimizzato per Raspberry.
F\'u sviluppato da Mike Thompson e Peter Green come progetto non affiliato alla compagnia Raspberry Pi
Fundation, pensato apposta per le basse prestazioni dei processori Advanced RISC Machine(ARM) montati sul
dispostivo.
La prima versione venne rilasciata nel 2012.


\section{Electron}
Electron \'e un Framework open source rilasciato per la prima volta nel 2013, ma la prima versione
stabile \'e uscita di recente. \'E disponibile sui sistemi operativi Window, MacOS e Linux ed \'e scritto
in C++ e Javascript. Il framework permette la creazione di interfacce grafiche (GUI) per
applicazioni cross platform, utilizzando teconologie gi\'e esistenti per lo sviluppo del backend e del frontend
(Javascript, NodeJS, V8 \cite{V8}).
All'avvio Electron inizializza una pagina in Chromium, che renderizza una pagina web, e un server
NodeJs che si occupa di trasmettere e interagire con il frontend.
Un'applicazione Electorn ha bisongo di 3 componenti principali:
\begin{itemize}
\item Il package.json, un file JSON, che deve contenere almeno il nome dell'applicazione,
la versione dell'applicazione creata, la descrizione di quest'ultima e il
 nome del file principale dell'applicazione (necessaria per l'avvio)
\begin{lstlisting}
{
  "name": "magicmirror",
  "version": "2.1.1",
  "description": "The open source smart platform",
  "main": "js/electron.js"
}
\end{lstlisting}
\item Un file html che contiene il template della pagina mostrata dall'applicazione
\item Un file JavaScript che contiene il codice di esecuzione dell'applicazione: creazione della finestra,
rendering della pagina, ecc...
\end{itemize}

\section{OpenCV}
OpenCV (Open Source Computer Vision Library) \'e una libreria software sviluppata intorno al 2000
utilizzata nell'abito della visione in tempo reale
da parte di una macchia per mezzo di input digitali, ottenuti tramite telecamera o fotocamera ad esempio.\\
La librearia \'e supportata per i linguaggi C++(linguaggio in cui \'e scritta e dunque di cui ha l'interfaccia primaria), C, Python e Java e
per diversi sistemi operativi, anche mobile.\\
OpenCV prede in input un'immagine o uno stream (come un video o una serie di immagini) e, utilizzando algoritmi
di visioning implementati al suo interno, riconosce oggetti o specifiche forme, inoltre se ne può aumentare l'efficienza
applicando algoritmi di Machine Learning per individuare e riconoscere oggetti specifici.

\section{Google Speech Recognition}
Negli ultimi anni Google ha ampliato sempre di più il suo catalogo per quanto rigarda
i servizi cloud e le web API. Google Speech To Text (o Google Speech Recognition) \'e un
servizio che, ricevendo in input un file audio o uno stream, ottenuto per mezzo di un
dispositivo di audio input, traduce il parlato in testo scritto.\\
L'API supporta oltre 110 lingue, e le librerie sono disponibili nei linguaggi C\#, GO, Java, Node.JS,
PhP, Python e Ruby, inoltre dispone di varianti:
\begin{itemize}
\item Una con interfaccia REpresentational State Transfer(REST), che comunica per mezzo di URI
\item Una con gRPC, un sistema di chiamata di procedura remota
\end{itemize}

\section{NodeJS}
NodeJS \'e una piattaforma open source, utilizzata per progettare il backend di un server web, sfruttando
il motore JavaScript V8 sviluppato da Google.\\
NodeJS esegue delle operazioni al verificarsi di uno specifico evento, che pu\'o essere un accesso ad una porta
del server, la richiesta di una pagina o l'invio di un form data.
Per gestire i pacchetti di questo linguaggio viene utilizzato NPM\cite{NPM}, una linea di comando che permette
di scaricare ed installare librerie private o pubbliche salvate su un database.
Durante lo stage \'e stata utlizzato il pattern architetturale Model-View-Controller(MVC), usando Express
come middleware, Mustache come Sistema di Template e MySQL come Database.

\subsection{Express}
Express \'e un framework per NodeJS che permette di creare applicazioni Web e API in JavaScript.
Il software viene usato per progettare il middleware di un server, che è composto di 3 parti importanti:
\begin{itemize}
\item il Routing, utilizzato per determinare come il server debba rispondere ad un determinato metodo di richiesta,
ricevuta sottoforma di URI.
\item i Modelli, creati per ogni entit\'a-oggetto che esiste all'interno del server, ad ognuno dei quali viene associato
un controller.
\item il Controller, che riceve la richiesta inoltrata dal Routing, nel quale sono descritte le operazioni da eseguire
(in base al modello cui fa riferimento) e la risposta da ritornare. Inoltre, il controller accede al database per apportare 
le modifiche ai Modelli.
\end{itemize}




\section{JavaScript}
Inzialmente usato per il client-side
