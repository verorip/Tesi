\chapter{Struttra del Magic Mirror}

Il Magic Mirror è un progetto ideato e sviluppato da Michael Teeuw, successivamente esteso nelle sue funzionalità da una moltitudine di utenti su GitHub.
Una prima versione è stata scritta completamente in Python, in seguito è stata creata una seconda versione nella quale si è preferito l'utilizzo di Electron,
che ha comportato una variazione di linguaggio, a favore di Javascript. In questo modo è stato possibile implementare un'interfaccia esteticamente più gradevole
ed è stato possibile implementare un sistema per far comunicare i moduli fra di loro.
\\[2\baselineskip]
\section{Perchè Magic Mirror?}
L'idea dell'autore è nata rifacendosi allo specchio magico dell'omonima fiaba
scritta dai fratelli Grimm, La Bella Addormentata.\\
Il software viene mostrato attraverso un
comune monitor, trasmettendo immagini poste su uno sfondo completamente nero. Applicando sopra
una semplice pellicola a specchio (la quale da un lato permette di specchiarsi e dall'altro di vedere
attraverso) si crea un effetto particolare per cui una persona riesce a specchiarsi
e allo stesso tempo riesce a vedere le scritte o le immagini trasmesse dal monitor.
\\[2\baselineskip]
\begin{figure}[H]
    \includegraphics[width=1\textwidth, height=0.6\textheight]{magic_mirror}
    \caption{Magic Mirror by Michael Teeuw}
\end{figure}

\section{Avvio ed Escuzione}
Il Magic Mirror viene avviato tramite riga di comando di una shell: npm start, che va a ricercare
il file javascript principale indicato dal package.json.
All'avvio viene eseguito il codice di Electron che ha il compito di creare ed avviare un server per il backend,
il cui compito è di caricare tutte le strtture dati dello specchio:
\begin{itemize}
\item i Moduli, che sono le entità che permettono di creare e configurare applicazioni da agganciare al software principale.
\item i Node Helper, file "speciali" che servono come supporto esterno ai moduli, per mezzo dei quali si possono interfacciare le api di servizi
esterni al Magic Mirror.
\item ggg\\[2\baselineskip]
\end{itemize}
ed una finestra di chromium per il fron end.
