\chapter{Struttura del Magic Mirror}

Il Magic Mirror \`e un progetto ideato e sviluppato da Michael Teeuw, successivamente esteso nelle sue funzionalit\`a da una moltitudine di utenti su GitHub.
Una prima versione \`e stata scritta completamente in Python, mentre successivamente \`e stata creata una seconda versione nella quale si \`e preferito l'utilizzo di Electron,
che ha comportato una variazione di linguaggio, a favore di Javascript. In questo modo \`e stato possibile implementare un'interfaccia esteticamente pi\`u gradevole
e API pi\`u intuitive.
\\[2\baselineskip]

\section{Perch\`e Magic Mirror?}
L'idea dell'autore \`e nata rifacendosi allo specchio magico dell'omonima fiaba
scritta dai fratelli Grimm, La Bella Addormentata.\\
Il software viene mostrato attraverso un
comune monitor, trasmettendo immagini poste su uno sfondo completamente nero. Applicando sopra
una semplice pellicola a specchio (la quale da un lato permette di specchiarsi e dall'altro di vedere
attraverso) si crea un effetto particolare per cui una persona riesce a specchiarsi
e allo stesso tempo riesce a vedere le scritte o le immagini trasmesse dal monitor,
come mostrato in figura \ref{fig:MM}.
\\[2\baselineskip]
\begin{figure}[H]
    \includegraphics[width=1\textwidth, height=0.6\textheight]{magic_mirror}
    \caption{Magic Mirror by Michael Teeuw}
    \label{fig:MM}
\end{figure}

\section{Avvio ed Escuzione}
Il Magic Mirror viene avviato tramite il comando \textit{npm start}, il quale esegue
il codice di un file Javascript, il cui percorso viene indicato nel documento \textit{package.json}.
Il primo file contiene il codice di Electron, che si occupa della creazione di una nuova finestra usata per rappresentare l'interfaccia mostrata dal browser.
Quest'ultimo contiene a sua volta i Document Object Model (DOM) e il codice dell'applicazione, ovvero il core dello specchio.
Quest'ultima carica tutte le strutture dello specchio, che sono le seguenti:
\begin{itemize}
\item gli end-point per le applicazioni, che sono le interfacce usate per leggere e caricare le applicazioni inserite nel Magic Mirror.
\item gli end-point per i Node Helper, per la gestione dei questi ultimi. Sono strutture opzionali usate per collegamenti
esterni al Magic Mirror (per esempio, con API di un servizio cloud). Ogni applicazione ha il proprio Node Helper con cui pu\`o comunicare tramite messaggi in modo
simile a come comunicano le applicazioni tra di loro.
\item un "Socket", entit\`a principale che definisce le funzioni e le metodologie per lo scambio dei messaggi tra le applicazioni e i rispettivi Node Helper.
\item un Logger, implementato per tenere i log dell'applicazione e degli evenutali errori. Usato pricipalmente per il debugging.
\item un file \textit{config.txt}, ovvero un file di configurazione dello specchio, nel quale sono segnate il nome e le coordinate per la posizione delle varie applicazioni
all'interno della pagina\\[2\baselineskip]
\end{itemize}
Inoltre viene inizializzato un server, il cui compito \`e quello di trasmettere la pagina renderizzata, con gli output delle varie applicazioni,
al browser precedentemente avviato.

\subsection{Il file di Configurazione}
Come gi\`a menzionato, il Magic Mirror carica un file di configurazione, che è composto dai seguenti campi:
\begin{itemize}
\item la porta del server
\item una whitelist, ovvero un IP oppure un range di IP che possono collegarsi allo specchio
\item la lingua principale del sistema
\item il formato del timer (12h o 24h)
\item unit\`a di misura usata (ad esempio, metrica)
\item una lista di applicazioni (in formato JSON) da caricare con la relativa posizione nella pagina. Per ogni applicazione deve necessariamente comparire il
nome e la posizione; opzionalmente si possono inserire un campo  \textit{header} e un campo \textit{config} specifico per l'applicazione.
\begin{lstlisting}
{
	"module": 'nome dell'applicazione',
	"position": 'la posizione dell'applicazione all'intenro dello specchio',
	"header": 'stampa sopra all'applicazione',
	"config": { opzioni varie in formato JSON }
}
\end{lstlisting}
\end{itemize}
La lingua, il formato del timer e l'unit\`a di misura usata sono strumenti messi a disposizione dal sistema per la creazione di un'applicazione
(ad esempio, il display di un orologio).

\section{Implementazione di un'applicazione}
La modifica del file di configurazione del Magic Mirror appena descritta serve per "notificare" la presenza delle applicazioni a quest'ultimo,
ma perch\`e possano funzionare \`e necessario che rispettino alcune specifiche regole.
Per inserire il codice dell'applicazione all'interno dello specchio \`e necessario creare una cartella con un nome identificativo dell'applicazione
nella directory \textit{Modules}.
Dentro la cartella appena creata devono essere inseriti:
\begin{itemize}
\item 1 file Javascript (JS), ovvero il documento principale con lo stesso nome della cartella appena creata. Contiene il codice dell'applicazione, il quale
conterr\`a a sua volta il codice per la creazione dei DOM
\item 1 file Cascading Style Sheets (CSS), per modificare l'estetica del DOM della relativa applicazione (opzionale)
\item 1 file node\_helper.js, che \`e il Node Helper associato alla specifica applicazione (opzionale)
\item Altri file necessari all'applicazione (immagini, JSON, etc)\\[1\baselineskip]
\end{itemize}
Il file Javascript principale dell'applicazione viene caricato per primo dal Magic Mirror, e consiste in una chiamata di funzione con i seguenti due parametri:
\begin{lstlisting}[language=JavaScript]
{
	Module.register("Nome dell'applicazione", { /* lista JSON di oggetti contenenti funzioni o variabili */});
}
\end{lstlisting}
Il primo parametro \`e una stringa contenente il nome dell'applicazione: quest'ultimo deve essere necessariamente uguale al nome della cartella e al nome
del file Javascript. Il secondo parametro, invece, \`e una lista JSON contenente funzioni o variabili (come in figura *da inserire*).
Le funzioni offerte dalle API sono:
\begin{itemize}
\item defaults: {}, una lista di variabili che possono essere richiamate all'interno di una qualsiasi funzione tramite il comando \textit{this.config.variabile}.
I valori di queste possono essere sovrascritti modificando il campo \textit{config} del relativo modulo nel file di configurazione del Magic Mirror.
\item start: function(){}, funzione che viene eseguita quando tutte le applicazioni dello specchio sono state caricate (ovvero quando sono stati creati tutti i relativi DOM)
\item getDom: function(){}, funzione che deve ritornare un DOM (un oggetto HTML contenente i dati da mostrare a schermo), creato tramite funzioni Javascript
\item getStyles: function() { return []}, funzione che ritorna un array di file (in formato CSS) usati per l'estetica del DOM. Possono essere nella cartella dell'applicazione
oppure ottenuti tramite link
\item getTranslations: function() {	return {en: "translations/en.json", de: "translations/de.json"}}, funzione per tradurre l'applicazione in pi\`u lingue;
se disponibile, viene caricata la traduzione in base alla lingua configurata nel software
\item getHeader: function() {	return this.data.header;}, funzione che stampa il campo \textit{header} della configurazione, con la possibilit\`a concaternarla
ad una stringa o ad un parametro
\item notificationReceived: function(notification, payload, sender) {}, funzione che serve per ricevere messaggi da altre applicazioni. Viene richiamata alla ricezione
\item socketNotificationReceived: function(notification, payload){}, funzione che serve per ricevere messaggi dal Node Helper della relativa applicazione\\[1\baselineskip]
\end{itemize}
Inoltre possono essere aggiunte delle funzioni necessarie all'applicazione implementandole con la stessa sintassi di quelle di default.
\\[1\baselineskip]
Il Node Helper, invece, viene caricato come libreria ed esportato insieme all'applicazione quando viene caricata dallo specchio. Per inizializzarlo sono
necessari i comandi:
\begin{lstlisting}[language=JavaScript]
{
  var NodeHelper = require("node_helper");
  module.exports = NodeHelper.create({/* lista JSON di oggetti contenenti funzioni*/});
}
\end{lstlisting}
A differenza del file Javascript principale, l'unica funzione messa a disposizione dall' API \`e:
\begin{lstlisting}[language=JavaScript]
{
  start: function() {}
}
\end{lstlisting}
mentre \`e possibile implementare le proprie funzioni con la stessa sintassi.

\section{Messaggistica del MM}
In precedenza \`e stato gi\`a accennato che il Magic Mirror implementa un meccanismo di messaggistica sfruttando un sistema di socket integrato,
utile per l'organizzazione e la moderazione delle applicazioni tramite l'utilizzo di funzioni messe a disposizione dall'API.
Le funzioni usate per ricevere i messaggi, come gi\`a detto precedentemente, sono: \textit{socketNotificationReceived(notification, payload)},
per la ricezione dei messagi da parte del Node Helper, e \textit{notificationReceived(notification, payload, sender)}, per la ricezione di
messaggi dalle altre applicazioni.
Per spedire messaggi vengono chiamate le funzioni \textit{sendSocketNotification(notification, payload)} e \textit{sendNotificationReceived(notification, payload, sender)},
i cui campi indicano:
\begin{itemize}
\item notification, l'indentificatore della notifica, una sorta di header al quale si pu\`o assegnare un qualunque oggetto
\item payload, il corpo del messaggio (opzionale)
\item sender, usato solo nei messaggi tra i moduli, dove viene riportato l'identificativo dell'applicazione che lo ha mandato\\[1\baselineskip]
\end{itemize}
\textit{sendNotificationReceived} a differenza del suo corrispettivo per il Node Helper, manda il messaggio in broadcast a tutte le applicazioni attive
sulla macchina, e deve essere filtrato dalle singole applicazioni in base al campo \textit{notification}.
