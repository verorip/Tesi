\chapter*{Conclusione}

Per riassumere il lavoro svolto, si ricorda che durante la prima parte dello stage sono state sviluppate due interfacce utente per il MM:
\begin{itemize}
\item Una vocale, che permette l'impartizione di comandi vocali validi. In questa parte si \`e dovuto affrontare il problema di interfacciarsi
ai servizi Google e di far comunicare correttamente il microfono con le API esposte, tramite il software SoX
\item Una touch, che permette l'impartizione di comandi validi tramite il movimento delle dita su una scheda touch.
Il problema riscontrato con questa interfaccia \`e stato implementare un canale di comunicazione tra il programma python che cattura i controlli
sulla scheda e il NodeHelper.
\end{itemize}
Nella seconda parte \`e stata implementata un'interfaccia web di configurazione remota
che permette di configurare pi\`u facilmente il MM e i suoi moduli.
I problemi riscontrati in questa parte sono stati:
\begin{itemize}
\item La corretta individuazione, visualizzazione e modifica dei file JSON
dei moduli, contenuti all'interno dei file locali del MM
\item L'utilizzo di una buona pratica di programmazione in Javascript per evitare fenomeni
che portassero ad una scrittura del codice molto confusionaria.\\[1\baselineskip]
\end{itemize}
Alcune parti di questi progetti sono ancora arcaici o poco sviluppati:
infatti un possibile sviluppo futuro delle interfacce utente \`e l'inserimento di un'unit\`a logica che possa validare i comandi
impartiti (sia vocali che touch), in modo pi\`u concreto rispetto ad un semplice  controllo di stringa.\\
Invece, per l'interfaccia web di configurazione, si pu\`o implementare un sistema per il riavvio automatico del software,
anzich\`e farlo manualmente, insieme ad una crittografia dei messaggi scambiati, in modo da inserire un livello di sicurezza.
