\chapter*{Introduzione}
\addcontentsline{toc}{chapter}{Introduzione}

Nello stage che ho svolto presso Lab121 è stato studiato il Magic Mirror, un software sviluppato da Michael Teeuw,
che gira sul calcolatore RaspberryPi, sul quale si sono svolte due principali attività:\\
-Lo sviluppo di applicazioni (o moduli) che si interfacciano con periferiche esterne o con api (italic)\\
-Lo sviluppo di un'interfaccia web che potesse modificare la configurazione del software senza dover accedere




Nel corso degli ultimi decenni con il perfezionamento delle teconologie e l'evoluzione dei
calcolatori, \'e risultato sempre pi\'u facile, sia dal punto di vista economico che dello spazio fisico,
ottenere dispositivi elettronici dall'alta portabilit\'a e dalla buona efficenza, di cui
alcuni hanno permesso anche lo sviluppo di molti progetti fatti in casa.
Questa evoluzione ha portato sempre di pi\'u alla "fusione" con le macchine, poterle controllare senza
dover necessariamente dover digitare dei caratteri o premere tasti,
basti guardare buona parte degli Smartphone moderni che permettono lo sblocco dello schermo con
il volto, oppure l'interazione con esso tramite sintesi vocale.
Al giorno d'oggi molti software di questo ambito relazione tra uomo-macchina
sono open source per permettere agli utenti di poterli evolvere e perfezionarli sempre di pi\'u.\\
Nella prima parte dello stage sono stati sviluppati dei moduli per il Magic Mirror, un software
di terze parti scritto in Javascript che gira su electron e NodeJS,
in diversi linguaggi di programmazione, che permettono l'interazione tra uomo-macchina per
mezzo di periferiche esterne.
Tra i moduli sviluppati vi sono il supporto per l'impartizione di comandi
tramite movimenti delle dita su una scheda TouchPad e un supporto per il riconoscimento vocale,
ovvero l'impartizione di comandi specifici per mezzo di un microfono.
\\[2\baselineskip]
Un altro aspetto che si \'e evoluto soprattuto negli ultimi anni nel campo
delle applicazioni e delle web app \'e l'implementazione di interfacce di gestione sempre pi\'u
intuitive e facili da comprendere. Capita spesso, infatti, che un'applicazione
venga utilizzata da terzi, anche senza che questi abbiano conoscenze informatiche,
se non vengono servite delle astrazioni a livello applicativo un utente, senza
tali conoscenze, non pu\'o gestire o configurare un'applicazione.
Nella seconda parte dello stage,
\'e stata sviluppata un'interfaccia web, la quale implementava un pannello di configurazione
il compito \'e di permettere di modificare la configurazione
dei moduli (anche quelli creati da terzi e inseriti successivamente) senza dover andare a modificare il
file in locale rendendo pi\'u facile aggiornare le features del dispositivo.
