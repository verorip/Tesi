\chapter*{Introduzione}
\addcontentsline{toc}{chapter}{Introduzione}

Nello stage che ho svolto presso Lab121 ho esteso il MagicMirror (abbreviato in \emph{MM}), un software di domotica sviluppato da Michael Teeuw,
che gira sul calcolatore portatile RaspberryPi\cite{Raspberry}, sul quale ho svolto due principali attivit\`a:\\
\begin{itemize}
\item lo sviluppo di applicazioni (detti anche \emph{Moduli}) mirate al controllo del
MagicMirror tramite l'impartizione di comandi vocali o input trasmessi dal movimento delle dita su
una periferica touchpad.\\
La scelta dello sviluppo di queste applicazioni \`e stata presa con l'obbiettivo di dotare il
MagicMirror di un'interfaccia uomo-macchina avanzata. Infatti, negli ultimi decenni, con lo sviluppo
di nuove teconologie, sono stati sviluppati ed evoluti metodi non tradizionali per il controllo di dispositivi.\\
Per esempio si possono osservare la maggior parte degli Smartphone moderni che permettono lo sblocco dello schermo con
il riconoscimento del volto oppure l'interazione con esso tramite sintesi vocale,
invece dell'utilizzo di tastiere analogiche o digitali.
Inoltre il progresso tecnolgico di calcolatori e periferiche sempre pi\`u piccoli ed economici, ha permesso
la nascita di una moltitudine di progetti casalinghi in ambito interazione uomo-macchina, contribuendo
alla crescita dei software di questo tipo, oggi reperibili in quantit\`a anche open source.\\[1\baselineskip]
\item lo sviluppo di un'interfaccia grafica web, la quale, tramite un pannello di amministrazione,
 permette di modificare il file di configurazione del MM e delle applicazioni che si appoggiano su di esso,
evitando cos\`i di dover accedere
a file potenzialmente critici ed eliminarne parti essenziali ad opera di utenti non esperti.
Infatti, oltre alle teconologie, un altro aspetto che si \`e evoluto negli ultimi anni
nel campo delle Applicazioni Web, \`e l'implementazione di interfacce grafiche di amministrazione
ad alto livello sempre pi\`u intuitive e facili da comprendere.
Capita spesso che, in assenza di queste interfacce pi\`u evolute, un utente inesperto sia
costretto ad utilizzare interfacce a basso livello (come strumenti di linea di comando o file
di configurazione), con il rischio di causare potenziali danni.\\[2\baselineskip]
\end{itemize}

Nei prossimi capitoli:
\begin{itemize}
\item Nel capitolo \ref{capitolo1} verranno trattati scopi e problemi affrontati nello stage
\item Nel capitolo \ref{capitolo2} verranno elencate e spiegate le teconologie che sono state utilizzate
durante lo stage
\item Nel capitolo \ref{capitolo3} verr\`a spiegata la struttura del MagicMirror
\item Nel capitolo \ref{capitolo4} verranno spiegati i moduli creati per il MagicMirror
\item Nel capitolo \ref{capitolo5} verr\`a esposto il funzionamento dell'interfaccia web di amministrazione
\end{itemize}
