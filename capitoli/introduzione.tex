\chapter*{Introduzione}
\addcontentsline{toc}{chapter}{Introduzione}

Nello stage che ho svolto presso Lab121 ho stato studiato il Magic Mirror, un software di domotica sviluppato da Michael Teeuw,
che gira sul calcolatore portatile RaspberryPi\cite{Raspberry}, sul quale ho svolto due principali attivit\`a:\\
\begin{itemize}
\item Lo sviluppo di applicazioni (detti anche moduli) che si interfacciano con periferiche esterne, o con api (italic), e il software
principale.
\item Lo sviluppo di un'interfaccia web che permette di modificarne la configurazione senza dover accedere
a file potenzialmente critici ed eliminarne parti essenziali ad opera di utenti non esperti.\\[1\baselineskip]
\end{itemize}
Per la prima parte dello stage sono state create applicazioni mirate al controllo del
Magic Mirror tramite l'impartizione di comandi vocali o input trasmessi dal movimento delle dita su
una periferica touchpad.\\
Questa scelta \`e stata presa prendendo atto che nel corso
degli ultimi decenni, con il perfezionamento delle teconologie e dei software, ha fatto nascere
una tendenza a controllare macchine o dispositivi con metodi sempre meno "diretti".
Per esempio si possono osservare la maggior parte degli Smartphone moderni che permettono lo sblocco dello schermo con
il riconoscimento del volto, oppure l'interazione con esso tramite sintesi vocale.
Inoltre, lo sviluppo dei calcolatori e delle periferiche sempre pi\`u piccoli ed economici ha permesso
la nascita di una moltitudine di progetti casalinghi con questo fine, contribuendo cos\'i
alla crescita dei software in ambito relazione uomo-machcina, oggi reperibili in quantità anche open source.
\\[2\baselineskip]
%e l'evoluzione dei
%calcolatori, \`e risultato sempre pi\`u facile, sia dal punto di vista economico che dello spazio fisico,
%ottenere dispositivi elettronici dall'alta portabilit\`a e dalla buona efficenza, di cui
%alcuni hanno permesso anche lo sviluppo di molti progetti fatti in casa.
%Questa evoluzione ha portato sempre di pi\`u alla "fusione" con le macchine, poterle controllare senza
%dover necessariamente dover digitare dei caratteri o premere tasti,
%basti guardare buona parte degli Smartphone moderni che permettono lo sblocco dello schermo con
%il volto, oppure l'interazione con esso tramite sintesi vocale.
%Al giorno d'oggi molti software di questo ambito relazione tra uomo-macchina
%sono open source per permettere agli utenti di poterli evolvere e perfezionarli sempre di pi\`u.\\
%Nella prima parte dello stage sono stati sviluppati dei moduli per il Magic Mirror, un software
%di terze parti scritto in Javascript che gira su electron e NodeJS,
%in diversi linguaggi di programmazione, che permettono l'interazione tra uomo-macchina per
%mezzo di periferiche esterne.
%Tra i moduli sviluppati vi sono il supporto per l'impartizione di comandi
%tramite movimenti delle dita su una scheda TouchPad e un supporto per il riconoscimento vocale,
%ovvero l'impartizione di comandi specifici per mezzo di un microfono.
Un altro aspetto che si \`e evoluto soprattutto negli ultimi anni nel campo
delle applicazioni e delle Applicazioni Web (chiamate anche webb app)
\`e l'implementazione di interfacce grafiche di gestione sempre pi\`u
intuitive e facili da comprendere. Capita spesso che un'applicazione
venga utilizzata da utenti privi di conoscenze informatiche, i quali, se mancasse
l'interfaccia di gestione, dovrebbero accedere direttamente ai file di configurazione
per modificarli, e potrebbero causare potenziali danni.\\
A tal fine, nella seconda parte dello stage,
\`e stata sviluppata un'interfaccia grafica web, la quale mostra un pannello di configurazione:
il compito \`e di permettere la corretta modifica dei file di configurazione delle applicazioni
e del Magic Mirror.
\\[2\baselineskip]
Nel capitolo 3 verr\`a spiegata la struttura del Magic Mirror.
